%% start of file `template.tex'.
%% Copyright 2006-2013 Xavier Danaux (xdanaux@gmail.com).
%
% This work may be distributed and/or modified under the
% conditions of the LaTeX Project Public License version 1.3c,
% available at http://www.latex-project.org/lppl/.


\documentclass[11pt,a4paper,sans]{moderncv}

% moderncv themes
\moderncvstyle{classic}
\moderncvcolor{blue}

% character encoding
\usepackage[utf8]{inputenc}
\usepackage[frenchb]{babel}

% adjust the page margins
\usepackage[scale=0.8]{geometry}

\usepackage{datetime}
\newdateformat{mydateformat}{\THEDAY~\monthname~\THEYEAR}

% personal data
\name{Arnaud}{Bletterer}
\address{2, Route de Forstheim}{67500 HAGUENAU}{FRANCE}
\phone[mobile]{06 64 32 44 28}
\phone[fixed]{03 69 02 09 52}
\email{arnaud.bletterer@etu.unistra.fr}
\homepage{abletterer.fr}
\photo[64pt][0.4pt]{Profil}

\begin{document}


%-----       letter       -----------------------------------------------------
% recipient data
\recipient{InSimo}
{c/o IHU de Strasbourg \\
1 Place de l'Hôpital,
67091 Strasbourg, France}
\date{\mydateformat\today}
\opening{Madame, Monsieur,}
\closing{Veuillez recevoir, Madame, Monsieur, l'expression de mes salutations
distinguées.}
\enclosure[Pièce jointe]{Curriculum Vit\ae{}}
\makelettertitle

Je suis étudiant en M2 Informatique, option Informatique et Sciences de
l'Image, à l'Université de Strasbourg. Je réalise actuellement mon stage de
Master au sein de l'équipe IGG du laboratoire ICube à Strasbourg, sous la
tutelle de Dominique BECHMANN, Isabelle CHARPENTIER et Pierre KRAEMER, autour
du sujet "Outils de déformation multidimensionnels". Le travail réalisé
pendant ce stage a permis d'aboutir à la mise en place d'une nouvelle méthode
de mélange d'outils de déformation spatiale permettant d'obtenir des
déformations conservant des propriétés de continuité.

Ce stage m'a permis de découvrir une première facette du monde de la
recherche. Celui-ci m'a réellement donné envie de continuer dans ce domaine.
Néanmoins, je trouve qu'une partie de la recherche manque d'applications et
qu'il n'y a pas de réelle concrétisation dans ce qui est réalisé. C'est
pourquoi je cherche à travailler dans un département de R&D au sein d'une
entreprise du domaine de l'informatique graphique. Les 

\makeletterclosing

\clearpage
%-----       resume       -----------------------------------------------------

\makecvtitle

\section{Scolarité}
\cventry{2012--2014}{Master Informatique, option
Informatique et Sciences de l'Image}{Université de Strasbourg}{STRASBOURG}{}{}
\cventry{2011--2012}{Licence Informatique}{Université de
Strasbourg}{STRASBOURG}{}{}
\cventry{2009--2011}{Diplôme Universitaire de Technologie Informatique}{IUT
Robert Schuman}{STRASBOURG}{}{}
\cventry{2009}{Baccalauréat Scientifique, option Sciences de l'Ingénieur}{LEGTI
Alphonse Heinrich}{HAGUENAU}{}{}

\section{Mémoire de Master}
\cvitem{\textbf{Sujet}}{\emph{Outils multidimensionnels de déformation}}
\cvitem{\textbf{Encadrants}}{Dominique BECHMANN, Isabelle CHARPENTIER, Pierre
KRAEMER}
\cvitem{\textbf{Description}}{Mise en place d'une méthode de mélange d'outils de
déformation}
\begin{itemize}
\item Etude des outils de déformation spatiale existants
\item Méthode de mélange d'outils de déformation conservant des propriétés de
continuité
\end{itemize}

\section{Expérience}
\subsection{Académique}
\cventry{2013 -- 3 mois}{Projet de développement}{Génération automatique de
cage}{Strasbourg, FRANCE}{}{}
\cventry{2013 -- 3 mois}{Travail d'Etude et de Recherche}{Multi-triangulation}
{Strasbourg, FRANCE}{}{}
\subsection{Professionnelle}
\cventry{2011 -- 3 mois}{Développeur JavaScript}{Dialog Insight}{Québec,
CANADA}{Stage de DUT}{}
\cventry{2009-2013}{Divers travaux en temps que saisonnier}{Schaeffler
FRANCE}{Haguenau, FRANCE}{}{}

% \section{Compétences Informatique}
% \cvitem{\textbf{Langages}}{C, C++, Java, C\#, Outils UNIX, HTML5, CSS3, PHP, JS,
% CGoGN}

\section{Langues}
\cvitem{Français}{Langue maternelle}
\cvitem{Anglais}{Avancé}
\cvitem{Allemand}{Scolaire}
\section{Intérêts}
\cvdoubleitem{\textbf{Sportifs}}{Athlétisme, Arts
Martiaux}{\textbf{Professionnels}}{~~Web, GPGPU}

\end{document}


%% end of file `template.tex'.
