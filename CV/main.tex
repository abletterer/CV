%% start of file `template.tex'.
%% Copyright 2006-2013 Xavier Danaux (xdanaux@gmail.com).
%
% This work may be distributed and/or modified under the
% conditions of the LaTeX Project Public License version 1.3c,
% available at http://www.latex-project.org/lppl/.


\documentclass[11pt,a4paper,sans]{moderncv}

% moderncv themes
\moderncvstyle{classic}
\moderncvcolor{blue}

% character encoding
\usepackage[utf8]{inputenc}
\usepackage[frenchb]{babel}

% adjust the page margins
\usepackage[scale=0.8]{geometry}

\usepackage{datetime}
\newdateformat{mydateformat}{\THEDAY~\monthname~\THEYEAR}

% personal data
\name{Arnaud}{Bletterer}
\address{2, Route de Forstheim}{67500 HAGUENAU}{FRANCE}
\phone[mobile]{06 64 32 44 28}
\phone[fixed]{03 69 02 09 52}
\email{arnaud.bletterer@etu.unistra.fr}
\homepage{abletterer.fr}
\photo[64pt][0.4pt]{Profil}

%-----       resume       -----------------------------------------------------
\begin{document}
\makecvtitle

\section{Scolarité}
\cventry{2012--2014}{Master Informatique, option
Informatique et Sciences de l'Image}{Université de Strasbourg}{STRASBOURG}{}{}
\cventry{2011--2012}{Licence Informatique}{Université de Strasbourg}{STRASBOURG}{}{}
\cventry{2009--2011}{Diplôme Universitaire de Technologie Informatique}{IUT
Robert Schuman}{STRASBOURG}{}{}
\cventry{2009}{Baccalauréat Scientifique, option Sciences de l'Ingénieur}{LEGTI
Alphonse Heinrich}{HAGUENAU}{}{}

\section{Mémoire de Master}
\cvitem{\textbf{Sujet}}{\emph{Outils multidimensionnels de déformation}}
\cvitem{\textbf{Encadrants}}{Dominique BECHMANN, Isabelle CHARPENTIER, Pierre KRAEMER}
\cvitem{\textbf{Description}}{Mise en place d'une méthode de mélange d'outils de
déformation}
\begin{itemize}
\item Etude des outils de déformation spatiale existants
\item Méthode de mélange d'outils de déformation conservant des propriétés de
continuité
\end{itemize}

\section{Expérience}
\subsection{Scientifique}
\cventry{2013 -- 3 mois}{Projet de développement}{Génération automatique de
cage}{Strasbourg, FRANCE}{}{}
\cventry{2013 -- 3 mois}{Travail d'Etude et de Recherche}{Multi-triangulation}
{Strasbourg, FRANCE}{}{}
\subsection{Professionnelle}
\cventry{2011 -- 3 mois}{Développeur JavaScript}{Dialog Insight}{Québec,
CANADA}{Stage de DUT}{}
\cventry{2009-2013}{Divers travaux en temps que saisonnier}{Schaeffler
FRANCE}{Haguenau, FRANCE}{}{}

\section{Compétences Informatique}
\cvitem{\textbf{Langages}}{C, C++, Java, C\#, Outils UNIX, HTML5, CSS3, PHP, JS,
CGoGN}

\section{Langues}
\cvitem{Français}{Langue maternelle}
\cvitem{Anglais}{Avancé}
\cvitem{Allemand}{Scolaire}
\section{Intérêts}
\cvdoubleitem{\textbf{Sportifs}}{Athlétisme, Arts Martiaux}{\textbf{Professionnels}}{~~Web, GPGPU}

\clearpage
%-----       letter       -----------------------------------------------------
% recipient data
\recipient{Ecole Doctorale "MSII"}
{Direction de la Recherche\\
Département Formation Doctorale\\
Collège des Ecoles Doctorales\\
46 Boulevard de la Victoire\\
67000 Strasbourg}
\date{\mydateformat\today}
\opening{Madame, Monsieur,}
\closing{Veuillez recevoir, Madame, Monsieur, l'expression de mes salutations
distinguées.}
\enclosure[Pièce jointe]{Curriculum Vit\ae{}}
\makelettertitle

Je suis étudiant en M2 Informatique, option Informatique et Sciences de l'Image,
à l'Université de Strasbourg. Je postule pour le sujet de thèse "Outils de
déformation multidimensionnels" proposé par Dominique BECHMANN, Professeure à
l'Université de Strasbourg et Directrice de l'équipe IGG du laboratoire ICube.

Je réalise actuellement mon stage de fin de Master au sein de l'équipe IGG du
laboratoire ICube, sous la tutelle de Dominique BECHMANN, Isabelle CHARPENTIER
et Pierre KRAEMER, autour du sujet "Outils de déformation multidimensionnels".
Ce stage a permis d'aboutir à la mise en place d'une méthode de mélange d'outils
de déformation spatiale permettant d'obtenir des déformations conservant des
propriétés de continuité. Ces résultats concluants laissent envisager un bon
avenir à la réussite d'une future thèse.

Je suis quelqu'un de très motivé, quand j'ai un objectif en vue je me donne tous
les moyens nécessaires pour y arriver. De plus, faire une thèse serait pour moi
le moyen de conclure mes études universitaires en appliquant l'ensemble des
connaissances que j'ai pu acquérir, et que j'acquérirai, afin d'améliorer la
qualité du travail que je fournirais.

\makeletterclosing

\end{document}


%% end of file `template.tex'.
