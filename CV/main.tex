%% start of file `template.tex'.
%% Copyright 2006-2013 Xavier Danaux (xdanaux@gmail.com).
%
% This work may be distributed and/or modified under the
% conditions of the LaTeX Project Public License version 1.3c,
% available at http://www.latex-project.org/lppl/.


\documentclass[11pt,a4paper,sans]{moderncv}

% moderncv themes
\moderncvstyle{classic}
\moderncvcolor{blue}

% character encoding
\usepackage[utf8]{inputenc}
\usepackage[frenchb]{babel}

% adjust the page margins
\usepackage[scale=0.8]{geometry}

\usepackage{datetime}
\newdateformat{mydateformat}{\THEDAY~\monthname~\THEYEAR}

% personal data
\name{Arnaud}{Bletterer}
\address{2, Route de Forstheim}{67500 HAGUENAU}{FRANCE}
\phone[mobile]{06 64 32 44 28}
\phone[fixed]{03 69 02 09 52}
\email{arnaud.bletterer@etu.unistra.fr}
\homepage{abletterer.fr}
\photo[64pt][0.4pt]{Profil}

\begin{document}


%-----       letter       -----------------------------------------------------
% recipient data
\recipient{Ecole Doctorale "MSII"}
{Direction de la Recherche\\
Département Formation Doctorale\\
Collège des Ecoles Doctorales\\
46 Boulevard de la Victoire\\
67000 Strasbourg}
\date{\mydateformat\today}
\opening{Madame, Monsieur,}
\closing{Veuillez recevoir, Messieurs, l'expression de mes salutations
distinguées.}
\enclosure[Pièce jointe]{Curriculum Vit\ae{}}
\makelettertitle

Je suis étudiant en M2 Informatique, option Informatique et Sciences de
l'Image, à l'Université de Strasbourg. Je postule pour le sujet de thèse "Du
nuage de points au maillage surfacique : modéliser et visualiser les données
3D massives".

Je réalise actuellement mon stage de Master au sein de l'équipe IGG du
laboratoire ICube à Strasbourg, sous la tutelle de Dominique BECHMANN,
Isabelle CHARPENTIER et Pierre KRAEMER, autour du sujet "Outils de déformation
multidimensionnels". Le travail réalisé pendant ce stage a permis d'aboutir à
la mise en place d'une nouvelle méthode de mélange d'outils de déformation
spatiale permettant d'obtenir des déformations conservant des propriétés de
continuité. L'expérience acquise lors de ce stage a confirmé ma volonté de
poursuivre en thèse dans le domaine des sciences informatiques.

Je n'ai pas encore travaillé dans le domaine de la construction de maillages
surfaciques à partir de nuages de points, néanmoins j'ai dû à de nombreuses
reprises explorer des sujets dont je ne connaissais rien avant. J'ai réussi à
m'adapter et à comprendre les problématiques de ces domaines, je ne vois donc
pas pourquoi ce ne serait pas le cas concernant ce sujet-ci. De plus le sujet
me semble réellement intéressant tant au niveau de l'objectif principal, que
sur des aspects plus généraux comme la compression de données et la
visualisation à distance.

\makeletterclosing

\clearpage
%-----       resume       -----------------------------------------------------

\makecvtitle

\section{Scolarité}
\cventry{2012--2014}{Master Informatique, option
Informatique et Sciences de l'Image}{Université de Strasbourg}{STRASBOURG}{}{}
\cventry{2011--2012}{Licence Informatique}{Université de
Strasbourg}{STRASBOURG}{}{}
\cventry{2009--2011}{Diplôme Universitaire de Technologie Informatique}{IUT
Robert Schuman}{STRASBOURG}{}{}
\cventry{2009}{Baccalauréat Scientifique, option Sciences de l'Ingénieur}{LEGTI
Alphonse Heinrich}{HAGUENAU}{}{}

\section{Mémoire de Master}
\cvitem{\textbf{Sujet}}{\emph{Outils multidimensionnels de déformation}}
\cvitem{\textbf{Encadrants}}{Dominique BECHMANN, Isabelle CHARPENTIER, Pierre
KRAEMER}
\cvitem{\textbf{Description}}{Mise en place d'une méthode de mélange d'outils de
déformation}
\begin{itemize}
\item Etude des outils de déformation spatiale existants
\item Méthode de mélange d'outils de déformation conservant des propriétés de
continuité
\end{itemize}

\section{Expérience}
\subsection{Académique}
\cventry{2013 -- 3 mois}{Projet de développement}{Génération automatique de
cage}{Strasbourg, FRANCE}{}{}
\cventry{2013 -- 3 mois}{Travail d'Etude et de Recherche}{Multi-triangulation}
{Strasbourg, FRANCE}{}{}
\subsection{Professionnelle}
\cventry{2011 -- 3 mois}{Développeur JavaScript}{Dialog Insight}{Québec,
CANADA}{Stage de DUT}{}
\cventry{2009-2013}{Divers travaux en temps que saisonnier}{Schaeffler
FRANCE}{Haguenau, FRANCE}{}{}

% \section{Compétences Informatique}
% \cvitem{\textbf{Langages}}{C, C++, Java, C\#, Outils UNIX, HTML5, CSS3, PHP, JS,
% CGoGN}

\section{Langues}
\cvitem{Français}{Langue maternelle}
\cvitem{Anglais}{Avancé}
\cvitem{Allemand}{Scolaire}
\section{Intérêts}
\cvdoubleitem{\textbf{Sportifs}}{Athlétisme, Arts
Martiaux}{\textbf{Professionnels}}{~~Web, GPGPU}

\end{document}


%% end of file `template.tex'.
