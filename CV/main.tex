%% start of file `template.tex'.
%% Copyright 2006-2013 Xavier Danaux (xdanaux@gmail.com).
%
% This work may be distributed and/or modified under the
% conditions of the LaTeX Project Public License version 1.3c,
% available at http://www.latex-project.org/lppl/.


\documentclass[11pt,a4paper,sans]{moderncv}

% moderncv themes
\moderncvstyle{classic}
\moderncvcolor{blue}

% character encoding
\usepackage[utf8]{inputenc}
\usepackage[frenchb]{babel}

% adjust the page margins
\usepackage[scale=0.8]{geometry}

\usepackage{datetime}
\newdateformat{mydateformat}{\THEDAY~\monthname~\THEYEAR}

% personal data
\name{Arnaud}{Bletterer}
\address{25 Avenue du Tapis Vert}{06220 VALLAURIS}{FRANCE}
\phone[mobile]{06 64 32 44 28}
\email{blettere@i3s.unice.fr}
\photo[64pt][0.4pt]{Profil}

\begin{document}

\makecvtitle
\vspace{-1cm}
\section{Scolarité}
\cventry{2012--2014}{Master Informatique, option
Informatique et Sciences de l'Image}{Université de Strasbourg}{}{}{}
\cventry{2011--2012}{Licence Informatique}{Université de
Strasbourg}{}{}{}
\cventry{2009--2011}{Diplôme Universitaire de Technologie Informatique}{IUT
Robert Schuman}{STRASBOURG}{}{}
\cventry{2009}{Baccalauréat Scientifique, option Sciences de l'Ingénieur}{LEGTI
Alphonse Heinrich}{HAGUENAU}{}{}

\section{Thèse de doctorat}
\cvitem{}{Commencée en Novembre 2014}
\cvitem{\textbf{Sujet}}{\emph{Du nuage de points au maillage surfacique : modéliser et visualiser les données 3D massives}}
\cvitem{\textbf{Encadrants}}{Marc ANTONINI et Frédéric PAYAN}
\cvitem{\textbf{Résumé}}{Dans le domaine de la numérisation 3D, les scanners 3D permettent de capturer la forme des objets qu'ils numérisent sous la forme de nuages de points. Aujourd'hui, ces outils sont capables d'acquérir plusieurs dizaines de millions de points en une seule acquisition. Ce flot de données est si volumineux et complexe qu'il est devenu difficile de manipuler et de visualiser les maillages reconstruits. Une solution à ce problème consiste alors à construire un maillage dit structuré ou semi-régulier, possédant une propriété multi-résolution intrinsèque, à partir des maillages irréguliers reconstruits. \newline
Ce travail s'intéresse à la reconstruction de maillages semi-réguliers directement à partir des nuages de points numérisés, sous la forme de données naturellement structurées fournies par les scanners 3D, les cartes de profondeur. \newline
Nous avons dans un premier temps étudié la possibilité de compresser et de visualiser des nuages de points massifs (plusieurs centaines de millions de points) de manière progressive en utilisant la structure régulière des cartes de profondeur (AFIG 2015, EI 2016). \newline
Ensuite, les recherches que nous avons faites nous ont montré qu'il était possible d'exploiter la structure des cartes de profondeur afin de reconstruire des maillages semi-réguliers à partir de celles-ci. 
Nous avons donc développé une méthode permettant de reconstruire, à différents niveaux de résolution, des acquisitions complexes et volumineuses (GRETSI 2017). \newline
Actuellement, nous travaillons sur plusieurs problématiques, à savoir la reconstruction d'un maillage à partir de la combinaison de plusieurs cartes de profondeur acquises depuis des points de vue différents, la subdivision de maillages grossiers reconstruits, à l'aide de différents filtres, pour obtenir une représentation progressive de ces maillages et finalement la compression de ces maillages, à l'aide de codeurs de maillages surfaciques, et l'évaluation de la fidélité en terme de distorsion géométrique par rapport à la surface numérisée.}

\section{Enseignement}
\subsection{2016/2017}
\cventry{16h - TP}{M1103}{Architecture des équipements informatiques}{IUT Nice Côte d'Azur}{}{}
\cventry{16h - TP}{M1207}{Bases de la programmation}{IUT Nice Côte d'Azur}{}{}
\cventry{32h - TP}{M2106}{Bases des services réseaux}{IUT Nice Côte d'Azur}{}{}
\subsection{2015/2016}
\cventry{16h - TP}{M1103}{Architecture des équipements informatiques}{IUT Nice Côte d'Azur}{}{}
\cventry{16h - TP}{M1207}{Bases de la programmation}{IUT Nice Côte d'Azur}{}{}
\cventry{32h - TP}{M2106}{Bases des services réseaux}{IUT Nice Côte d'Azur}{}{}

\section{Expérience}
\cventry{2014 -- 6 mois}{Stage de Master}{Outils multidimensionnels de déformation}{Université de Strasbourg}{}{}
\cventry{2013 -- 3 mois}{Projet de développement}{Génération automatique de cage}{Université de Strasbourg}{}{}
\cventry{2013 -- 3 mois}{Travail d'Etude et de Recherche}{Multi-triangulation}{Université de Strasbourg}{}{}
\cventry{2011 -- 3 mois}{Stage de DUT}{Dialog Insight}{Québec, CANADA}{}{}

\section{Langues}
\cvitem{Français}{Langue maternelle}
\cvitem{Anglais}{Avancé}
\cvitem{Allemand}{Scolaire}

\section{Communications}
\cventry{Novembre 2015}{Réunion GDR}{Imagerie Multi-vues : de l'acquisition à la projection}{}{}{}
\cventry{Mai 2015}{Réunion GDR}{AC3D - De l'acquisition à la compression des objets 3D}{}{}{}

\section{Publications}
\cventry{2017}{De la carte de profondeur au maillage surfacique : reconstruction de scènes 3D complexes}{A. Bletterer, F. Payan, M. Antonini, A. Meftah}{GRETSI 2017}{(à paraître)}{}
\cventry{2016}{Point Cloud Compression using Depth Maps}{A. Bletterer, F. Payan, M. Antonini, A. Meftah}{Electronic Imaging 2016 (EI 2016)}{}{}
\cventry{2015}{Utilisation de cartes de profondeur pour la visualisation de nuages de points}{A. Bletterer, F. Payan, M. Antonini}{28e journées de l'Association Française d'Informatique Graphique (AFIG 2015)}{}{}

\end{document}


%% end of file `template.tex'.
